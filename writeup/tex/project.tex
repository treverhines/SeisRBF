\documentclass[12pt]{article}

\usepackage{amssymb,amsmath}
\usepackage[margin=1.0in]{geometry}
\usepackage{fancyhdr} % required for custom header
\usepackage{graphicx}
\usepackage{listings}
\usepackage{courier}
\usepackage[usenames,dvipsnames]{color}

%set up the header
\pagestyle{fancy}
\lhead{Trever Hines}
\chead{Lab 03: Burgers' Equation on a Periodic Domain}
\rhead{\today}

\setlength{\headheight}{15pt}
\renewcommand\headrulewidth{1.0pt} % Size of the header rule

%% Code inclusion
%%-------------------------------------------------------------------------------
% create a function that import a python script
\lstloadlanguages{Python}
%        frame=single,

\definecolor{blay}{rgb}{0.93,0.93,1.0}
\lstset{language=Python,
        backgroundcolor=\color{blay}, 
        basicstyle=\footnotesize,
        keywordstyle=[1]\color{Blue}\bf,
        commentstyle=\color{Red},
        stringstyle=\color{ForestGreen},
        showstringspaces=false,
        numbers=left,
        numberstyle=\tiny,
        numbersep=3pt,
        stepnumber=2}


\newcommand{\pythonscript}[2]{\begin{itemize}
                              \item[]\lstinputlisting[caption=#2,label=#1]{#1.py}
                              \end{itemize}}

%% Title
%%------------------------------------------------------------------------------
\title{	
AOSS 555 Final Project:\\
Modeling Seismic Wave Propagation with Radial Basis Functions\\
\author{Trever Hines}
\rule{\headwidth}{1.0pt}
}


\begin{document}
\maketitle
\section*{Introduction}
Solve the problem
\begin{equation}\label{Problem}
  u_t = F(u),
\end{equation}
where
\begin{equation}\label{F}
  F(u) = \nu u_{xx} - uu_x,
\end{equation}
u is $2\pi$ periodic, $\nu=1/10$, and the initial condition are
\begin{equation}\label{IC}
  u(x,0) = 1 - \sin(x).
\end{equation}
Use the Fast Fourier Transform and an explicit time-marching method to
integrate from t=0 to t=2.  Present graphs illustrating
\begin{enumerate}
\item the evolution of the Fourier coefficients with time and
\item the evolution of u(x,t) with time.
\end{enumerate}

\section*{Solution}

I solve eq. (\ref{Problem}) by first discretizing the time domain into
$M$ time steps as
\begin{equation}
  t_j = \frac{2j}{M}, \quad j=\{0,1,...,M-1\}
\end{equation} 
and then I find $u(x,t_{j+1})$ with an explicit Runge-Kutta scheme.
For each iteration, eq. (\ref{F}) is evaluated at $u(x,t_j)$ as
descibed in the following paragraph.  

I approximate $u(x,t_j)$ with a complex exponential series containing
$N$ terms:
\begin{equation}\label{series}
  u(x,t_j) \approx \sum_{k=-N/2}^{N/2-1}\alpha_{jk}e^{ikx}.
\end{equation}
The choice of exponential basis functions ensures that the $2\pi$
periodic condition is satisfied.  I define my $N$ collocation points
as
\begin{equation}
  x_n = \frac{2\pi n}{N}, \quad n = \{0,1,...,N-1\},
\end{equation}
and then find $\alpha_{jk}$ for the current time step by making use of the
discrete Fourier transform:
\begin{equation}\label{dft}
  \alpha_{jk} = \mathrm{DFT}[u(x_n,t_j)]_k =
  \frac{1}{N}\sum_{n=0}^{N-1}u(x_n,t_j)e^{-ikx_n}, \quad
  k=\{-N/2,...,N/2-1\}.
\end{equation}  
I then evaluate eq. (\ref{F}) substituting $u$ with the series in
eq. (\ref{series}) and using the coefficients found from
eq. (\ref{dft}). For computational efficiency, the derivatives inside
eq. (\ref{F}) are evaluated in the Fourier domain. Namely, I use the
properties
\begin{equation}
  \mathrm{DFT}[u_{x}(x_n,t_j)]_k = (ik)\mathrm{DFT}[u(x_n,t_j)]_k = (ik)\alpha_{jk}
\end{equation}   
and
\begin{equation}
  \mathrm{DFT}[u_{xx}(x_n,t_j)]_k = (ik)^2\mathrm{DFT}[u(x_n,t_j)]_k = (ik)^2\alpha_{jk}
\end{equation}   
to evaluate eq. (\ref{F}) as
\begin{equation}
  F(u(x_n,t_j)) = \mathrm{IDFT}[\nu(ik)^2\alpha_{jk}]_n -
  u(x_n,t_j)\mathrm{IDFT}[(ik)\alpha_{jk}]_n,
\end{equation}   
where IDFT is the inverse discrete Laplace transform, which I define as
\begin{equation}\label{ift}
  u(x_n,t_j) = \mathrm{IDFT}[\alpha_{jk}]_n =
  \sum_{k=-N/2}^{N/2-1}\alpha_{jk}e^{ikx_n}, \quad
  n=\{0,1,...,N-1\}.
\end{equation}   
In total, evaluating eq. (\ref{F}) requires three Fourier transforms
and the computational cost for each time step is $O(N\log N)$ when
using the Fast Fourier Transform algorithm.  

The procedure described above is demonstrated in the below Python script.
%\pythonscript{Lab03_snippet}{Lab03.py} 

\section*{Results}
The solution for $u(x,t)$ using $M=1000$ and $N=200$ is shown in
figure 1.  As time progresses, the initial sine wave moves in the
positive $x$ direction while also becoming steeper on the leeward
side.  The amplitude of the wave also decreases over time as $u(x,t)$
approaches its steady state value of 1.      

Figure 2 shows the magnitude of the Fourier coefficients,
$\alpha_{jk}$, over time.  The coefficients are spectrally accurate
throughout the time interval from 0 to 2.  However, the amplitude of
the high frequency coefficients increases over time and it is likely
that the solution for $u(x,t)$ would become unstable if I continued
time stepping much further past $t=2$.


\end{document}

