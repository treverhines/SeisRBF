\documentclass[12pt]{article}

\usepackage{amssymb,amsmath}
\usepackage[margin=1.0in]{geometry}
\usepackage{fancyhdr} % required for custom header
\usepackage{graphicx}
\usepackage{listings}
\usepackage{courier}
\usepackage{natbib}
\usepackage{algorithm}
\usepackage{algorithmic}
\usepackage[usenames,dvipsnames]{color}

%set up the header
\pagestyle{fancy}
\lhead{Trever Hines}
\chead{Modeling Seismic Waves}
\rhead{\today}

\setlength{\headheight}{15pt}
\renewcommand\headrulewidth{1.0pt} % Size of the header rule

%% Code inclusion
%%-------------------------------------------------------------------------------
% create a function that import a python script
\lstloadlanguages{Python}

\definecolor{blay}{rgb}{0.93,0.93,1.0}
\lstset{language=Python,
        backgroundcolor=\color{blay}, 
        basicstyle=\footnotesize,
        keywordstyle=[1]\color{Blue}\bf,
        commentstyle=\color{Red},
        stringstyle=\color{ForestGreen},
        showstringspaces=false,
        numbers=left,
        numberstyle=\tiny,
        numbersep=3pt,
        stepnumber=2}

\newcommand{\pythonscript}[2]{\begin{itemize}
                              \item[]\lstinputlisting[caption=#2,label=#1]{#1.py}
                              \end{itemize}}

%% Title
%%------------------------------------------------------------------------------
\title{	
AOSS 555 Final Project:\\
Modeling Seismic Wave Propagation with Radial Basis Functions\\
\author{Trever Hines}
\rule{\headwidth}{1.0pt}
}


\begin{document}
\maketitle
\section*{Introduction}
 
In recent years seismologists have become interested in using
observable seismic waveforms to image the elastic properties of the
Earth's interior.  We use the term 'seismic waveform' to refer the the
displacement history that can be observed on a seismogram over the
course of hundreds of seconds following an earthquake.  As with any
inverse problem, imaging the elastic properties of the earth using
seismic waveforms requires one to be able to solve the forward
problem, that is, to compute the displacements that would be predicted for
a given realization of the Earth's interior.  Researchers have used
various numerical methods to solve this forward problem including
finite difference methods, finite element methods, and spectral
element methods (see \citet{F2011} for a review of numerical methods
in seismology).  In this paper We will explore the potential for using
radial basis functions (RBFs) to solve for the time dependent
displacements throughout the Earth following an earthquake.
 
One reason why RBFs are appealing is because they offer freedom in
choosing the collocation points.  In global seismology, one would
ideally have a high clustering of collocation points at the
core-mantle boundary, which is an interfaces where the elastic
properties suddenly and drastically change.  Additionally, the crust
and the core have relatively low velocities and would require a higher
density of nodes in accordance with the CFL criteria.  RBFs seem to be
appropriate for handling such heterogeneity in the Earth.

Another great advantage to RBFs is that they can be used to solve a
problem on an arbitrarily complicated domain.  Although the domain in
global seismology is typically taken to be a sphere or a disk, more
complicated geometries can arise in regional scale seismology problems
where topography may need to be considered.

\section*{Problem Formulation}
We can apply Newton's second law of motion to an infinitesimal parcel
within a continuum to obtain displacements, $\boldsymbol{u}$,
at any point $\boldsymbol{x}$ as
\begin{equation}\label{MomentumEquation}
  \rho\frac{\partial^2 u_i}{\partial t^2}  = \frac{\partial \tau_{ij}}{\partial x_j} + f_i,
\end{equation}
where $\boldsymbol{\tau}$ and $\boldsymbol{f}$ are the stress tensor
and body force per unit volume acting on point $\boldsymbol{x}$,
respectively, and $\rho$ is the density at point
$\boldsymbol{x}$. Note that We have adopted Einstein notation where
summations are implied by repeated indices.  We can express
eq. (\ref{MomentumEquation}) in terms of $\boldsymbol{u}$ by using the
definition of infinitesimal strain,
\begin{equation}\label{Strain}
  \varepsilon_{ij} = \frac{1}{2}\left(\frac{\partial u_j}{\partial x_i} + 
                                \frac{\partial u_i}{\partial x_j}\right),
\end{equation}
and Hook's law in an isotropic medium,
\begin{equation}\label{HooksLaw}
  \tau_{ij} = \lambda\delta_{ij}\varepsilon_{kk} + 2\mu\varepsilon_{ij},
\end{equation}
where $\lambda$ and $\mu$ are Lam\'e parameters which we generally
assume to be heterogeneous in the Earth.  The differential equation
being solved is then
\begin{align}\label{DiffEq}
  \rho\frac{\partial^2 u_i}{\partial t^2} &= 
    \frac{\partial}{\partial x_j}\left(\lambda\delta_{ij}\frac{\partial u_k}{\partial x_k} +
    \mu\left(\frac{\partial u_j}{\partial x_i} + \frac{\partial
    u_i}{\partial x_j}\right)\right) + f_i\\\nonumber
  &= 
    \frac{\partial}{\partial x_i}\left(\lambda\frac{\partial u_k}{\partial x_k}\right) +
    \frac{\partial}{\partial x_j}\left(\mu\left(\frac{\partial u_j}{\partial x_i} + \frac{\partial
    u_i}{\partial x_j}\right)\right) + f_i\\\nonumber
  &= 
    \frac{\partial \lambda}{\partial x_i}
    \frac{\partial u_k}{\partial x_k} +
    \lambda\frac{\partial^2 u_k}{\partial x_i x_k} +
    \frac{\partial \mu}{\partial x_j}\frac{\partial u_j}{\partial x_i} +
    \frac{\partial \mu}{\partial x_j}\frac{\partial u_i}{\partial x_j} +
    \mu\frac{\partial^2 u_j}{\partial x_j \partial x_i} + 
    \mu\frac{\partial^2 u_i}{\partial x_j \partial x_j} + f_i.
\end{align}
We impose free-surface boundary conditions, which means that
\begin{equation}\label{BoundaryCondition}
  \boldsymbol{\tau}\cdot\boldsymbol{\hat{n}} = 0
\end{equation}
for all $\boldsymbol{x}$ on the surface of the Earth, which have
surface normal vectors $\boldsymbol{\hat{n}}$.  We can write the
boundary conditions on $\boldsymbol{u}$ more explicitly as
\begin{equation}\label{BoundaryConditionExplicit}
  \hat{n}_i\lambda\frac{\partial u_k}{\partial x_k} +
  \hat{n}_j\mu\left(\frac{\partial u_j}{\partial x_i} +
  \frac{\partial u_i}{\partial x_j}\right) = 0.
\end{equation}

Although the equation that we are trying to solve looks daunting, if
we consider a homogeneous medium, neglect body forces, and assume
that displacement are in one direction and propagate in one direction,
then we can see that eq. (\ref{DiffEq}) reduces to the one dimensional
wave equation. In particular, if displacements are in the direction of
wave propagation then eq. (\ref{DiffEq}) becomes
\begin{equation}\label{1DS}
  \frac{\partial^2 u}{\partial t^2} = \frac{\lambda + 2\mu}{\rho}\frac{\partial^2 u}{\partial x^2},
\end{equation}
where waves propagate at the P wave velocity, $\sqrt{(\lambda +
  2\mu)/\rho}$. If displacements are perpendicular to the direction of
wave propagation then eq. (\ref{DiffEq}) reduces to
\begin{equation}\label{1DP}
  \frac{\partial^2 u}{\partial t^2} = \frac{\mu}{\rho}\frac{\partial^2 u}{\partial x^2},
\end{equation}
where waves propagate at the S wave velocity, $\sqrt{\mu/\rho}$.  

When solving eq. (\ref{DiffEq}), we assume the Earth is initially
stationary and that all displacements are caused by an earthquake.
Earthquakes are a sudden dislocation in the displacement field along a
fault plane and so it seems reasonable that earthquakes should be
imposed in this model by setting appropriate displacement boundary
conditions on a prescribed fault plane.  However, imposing a
dislocation in the displacement field would undoubtedly cause
numerical troubles.  Fortunately, we can instead represent an
earthquake as a collection of slightly offset point force couples,
$\boldsymbol{F}^{(i,j)}$ \citep{AR2002}.  All nine unit force
couples needed to represent a dislocation in three dimensional space
are shown in figure 1. Approximating a dislocation as point force
couples is reasonable if one is concerned with displacements far away
from the source, i.e. seismic waves on a global scale.  The finite
length of a fault rupture must be considered if one is interested in
near field displacements resulting from an earthquake.  Even so,
seismologists routinely represent earthquakes in terms of point force
couples using what is known as a moment tensor, $\boldsymbol{M}$.  The
components of the moment tensor are
\begin{equation}\label{MomentTensorComponents}
  M_{ij} = |\boldsymbol{F}^{(i,j)}|d,
\end{equation}
where $d$ is the offset between the force couples, which would ideally
be infinitesimally small.  To give the moment tensor more tangible
meaning, we can consider an isotropic compression source, such as a
subsurface explosion, where the force couples needed to describe the
source are $\boldsymbol{F^{(1,1)}}$, $\boldsymbol{F^{(2,2)}}$, and 
$\boldsymbol{F^{(3,3)}}$, where each couple has the same magnitude, $M_o/d$.
The moment tensor that describes the source is then
\begin{equation}
  \boldsymbol{M} = M_o\boldsymbol{I}.
\end{equation}
For our numerical simulation, we imposed an earthquake with moment
tensor
\begin{equation}
  \boldsymbol{M} = 
  \begin{vmatrix}
  0&10^{20}&0 \\
  10^{20}&0&0 \\
  0&0&0 \\
  \end{vmatrix} N m,
\end{equation}
which is equivalent to a Mw7.3 earthquake on a fault that is aligned
with either the first or second coordinate axis.  

We can represent the body force density for each force couple (without
implied summations) as
\begin{equation}\label{ForceCouples}
 \boldsymbol{f}^{(i,j)} =
 \frac{M_{ij}\boldsymbol{\hat{e}}_i}{d}
 \left(\delta\left(\boldsymbol{x} -
 \frac{d\boldsymbol{\hat{e}}_j}{2}\right) - \delta\left(\boldsymbol{x} +
 \frac{d\boldsymbol{\hat{e}}_j}{2}\right)\right),
\end{equation}
where $\mathbf{\hat{e}}_i$ denotes a unit basis vector in direction
$i$ and $\delta$ is a three dimensional delta function, which can be
taken to be a three dimensional Gaussian function with an infinitely
small width.  We can then represent the forcing term in 
$\ref{DiffEq}$ as 
\begin{equation}
 \boldsymbol{f} = \sum_{ij}  
 \frac{M_{ij}\boldsymbol{\hat{e}}_i}{d}\left(\delta\left(\boldsymbol{x} -
 \frac{d\boldsymbol{\hat{e}}_j}{2}\right) - \delta\left(\boldsymbol{x} +
 \frac{d\boldsymbol{\hat{e}}_j}{2}\right)\right).
\end{equation}
Although we circumvented to troubles of imposing a dislocation in the
displacement field, we now face the difficulty of numerically
representing a delta function.  Indeed, the accuracy of our solution
will come down to our ability to accurately represent a point force,
which requires a very high density of collocation points.  Using a
desktop computer with 16 GB of RAM, we are only able to accurately
represent a delta function as a 100 km wide Gaussian function, which
is certainly not infinitesimally small compared to the Earth but it is
the best we can do.  

For the solutions presented in this paper, we assume that all energy
is instantaneously released during an earthquake and $\boldsymbol{f}$
has no time dependence.  This assumption may not be reasonable because
large earthquakes could last for several minutes, which is on the
timescale of seismic wave propagation.  

\begin{figure}
\includegraphics[scale=0.4]{figures/MomentTensor}
\centering
\caption{Force couples, $\boldsymbol{F}^{(i,j)}$, that can
  be used to represent any type of dislocation on a fault (taken from
  \citet{AR2002})}

\end{figure}

\section*{Spectral Discretization}
We will assume, for computational tractability but without loss of
generality, that the domain is a two dimensional cross-section of the
Earth and displacements are only in the plane of that cross-section.
After expanding the summation notation, eq. (\ref{DiffEq}) can then be
written as
\begin{equation}
\begin{vmatrix}
  \partial^2 u_1/\partial t^2\\ \\
  \partial^2 u_2/\partial t^2
\end{vmatrix} =
\begin{vmatrix}
 \partial_1\left((\lambda + 2\mu)\partial_1 u_1\right) +  
 \partial_2\left(\mu\partial_2 u_1\right) + 
 \partial_1\left(\lambda\partial_2 u_2\right) +  
 \partial_2\left(\mu\partial_1 u_2\right)\\ \\
 \partial_1\left(\mu\partial_2 u_1\right) +  
 \partial_2\left(\lambda\partial_1 u_1\right) + 
 \partial_2\left((\lambda + 2\mu)\partial_2 u_2\right) +  
 \partial_1\left(\mu\partial_1 u_2\right) 
\end{vmatrix} + 
\begin{vmatrix}
  f_1\\ \\
  f_2
\end{vmatrix},
\end{equation}
where we use $\partial_i$ to denote a derivative with respect to
$x_i$.  Likewise, the boundary conditions can be written as
\begin{equation}
\begin{vmatrix}
  0\\ \\
  0
\end{vmatrix} =
\begin{vmatrix}
 \hat{n}_1(\lambda + 2\mu)\partial_1 u_1 +  
 \hat{n}_2\mu\partial_2 u_1 + 
 \hat{n}_1\lambda\partial_2 u_2 +  
 \hat{n}_2\mu\partial_1 u_2\\ \\
 \hat{n}_1\mu\partial_2 u_1 +  
 \hat{n}_2\lambda\partial_1 u_1 + 
 \hat{n}_2(\lambda + 2\mu)\partial_2 u_2 +  
 \hat{n}_1\mu\partial_1 u_2
\end{vmatrix}. 
\end{equation}

We chose to solve eq. (\ref{DiffEq}) in the spatial dimension using
multiquadratic radial basis functions, which are defined as
\begin{equation}\label{RBF}
  \phi(\boldsymbol{x};\boldsymbol{x_0}) = \sqrt{1 + (\epsilon|\boldsymbol{x} - \boldsymbol{x}_0|)^2}. 
\end{equation}
In the above equation, $\epsilon$ is the shape parameter and
$\boldsymbol{x}_0$ is the center of the RBF.  We chose to use
multiquadratic radial basis functions because interpolating a function
using $\phi(\boldsymbol{x};\boldsymbol{x_0})$ has the desirable trait of being
equivalent to piece-wise linear interpolation for $\epsilon \to
\infty$ and polynomial interpolation for $\epsilon \to 0$ at least for
one-dimensional problems \citep{D2002}.  This suggests that a wide
range of values for $\epsilon$ should produce reasonably accurate
results because either end-member is a perfectly valid means of
interpolation.  This is in contrast to using RBFs that decay to zero
as $|\boldsymbol{x - x_o}| \to \infty$, where the interpolant for
$\epsilon \to \infty$ is essentially useless.  We follow the common
practice of making $\epsilon$ for each RBF proportional to the
distance to the nearest RBF center.  We did not rigorously search for
an optimal proportionality constant; however, we note that we tend to
get unstable or obviously inaccurate solutions for proportionality
constants greater than $2$ and less than $0.4$.  In particular, small
values of epsilon tend to produce instabilities at sharp material
interface or at the surface.  large values of epsilon create
large scale, unrealistic undulations. For the solution presented in
this paper, we used a proportionality constant of 1, that is, we set
$\epsilon$ equal to the distance to the nearest neighbor.

We approximate the solution for displacements as 
\begin{equation}
  u_1 \approx \sum_{j=1}^{N}\alpha_{j}(t)\phi_j(\boldsymbol{x};\boldsymbol{x_j})
\end{equation}
and
\begin{equation}
  u_2 \approx \sum_{j=1}^{N}\beta_{j}(t)\phi_j(\boldsymbol{x};\boldsymbol{x_j}),
\end{equation}
which can also be written as 
\begin{equation}
\begin{vmatrix}
  u_1 \\
  u_2 
\end{vmatrix} \approx 
\boldsymbol{G}
\begin{vmatrix}
  \boldsymbol{\alpha}\\
  \boldsymbol{\beta}
\end{vmatrix}
\end{equation}
where $\boldsymbol{G}$ is given by
\begin{equation}
\boldsymbol{G} = 
\begin{vmatrix}
\boldsymbol{\phi}&\boldsymbol{0}\\
\boldsymbol{0}&\boldsymbol{\phi}
\end{vmatrix}
\end{equation}
and we are no longer explicitly writing the time and space dependence.
We can then express acceleration in terms of the spectral
coefficients, $\boldsymbol{\alpha}$ and $\boldsymbol{\beta}$, as
\begin{equation}\label{Acceleration}
\begin{vmatrix}
  \partial^2 u_1/\partial t^2\\
  \partial^2 u_2/\partial t^2
\end{vmatrix} =
\boldsymbol{H}
\begin{vmatrix}
  \boldsymbol{\alpha}\\
  \boldsymbol{\beta}
\end{vmatrix} + 
\begin{vmatrix}
  f_1\\
  f_2
\end{vmatrix},
\end{equation}
where
\begin{equation}
\boldsymbol{H} = 
\begin{vmatrix}
 \partial_1\left((\lambda + 2\mu)\partial_1 \boldsymbol{\phi}\right) +  
 \partial_2\left(\mu\partial_2 \boldsymbol{\phi}\right)&
 \partial_1\left(\lambda\partial_2 \boldsymbol{\phi}\right) +  
 \partial_2\left(\mu\partial_1 \boldsymbol{\phi}\right)\\
 \partial_1\left(\mu\partial_2 \boldsymbol{\phi}\right) +  
 \partial_2\left(\lambda\partial_1 \boldsymbol{\phi}\right)&
 \partial_2\left((\lambda + 2\mu)\partial_2 \boldsymbol{\phi}\right) +  
 \partial_1\left(\mu\partial_1 \boldsymbol{\phi}\right) 
\end{vmatrix}.
\end{equation}
The boundary conditions can also be expressed in terms of the spectral
coefficients as
\begin{equation}
\begin{vmatrix}
  0\\
  0
\end{vmatrix} =
\boldsymbol{B}
\begin{vmatrix}
  \boldsymbol{\alpha}\\
  \boldsymbol{\beta}
\end{vmatrix},
\end{equation}
where
\begin{equation}
\boldsymbol{B} = 
\begin{vmatrix}
 \hat{n}_1(\lambda + 2\mu)\partial_1 \boldsymbol{\phi} +  
 \hat{n}_2\mu\partial_2 \boldsymbol{\phi} &
 \hat{n}_1\lambda\partial_2 \boldsymbol{\phi} +  
 \hat{n}_2\mu\partial_1 \boldsymbol{\phi}\\
 \hat{n}_1\mu\partial_2 \boldsymbol{\phi} +  
 \hat{n}_2\lambda\partial_1 \boldsymbol{\phi} &
 \hat{n}_2(\lambda + 2\mu)\partial_2 \boldsymbol{\phi} +  
 \hat{n}_1\mu\partial_1 \boldsymbol{\phi}
\end{vmatrix}.
\end{equation}
At this point, all the information in $\boldsymbol{G}$,
$\boldsymbol{H}$, $\boldsymbol{B}$, and $\boldsymbol{f}$ are given by
the problem description or are user defined variables.  Finding the
spectral coefficients for $\boldsymbol{u}$ is then straightforward.
We perform the time integration with a leapfrog method and our
procedure for solving eq. (\ref{DiffEq}) is as follows:
\begin{enumerate}
\item Form the matrix $\boldsymbol{J}$, which consists of
  $\boldsymbol{B}$ evaluated at the boundary collocation points and
  $\boldsymbol{G}$ evaluated at the interior collocation points. We
  use the collocation points as the RBF centers.
\item Form the vector $\boldsymbol{r}^0$, which consists of zeros for
  the boundary collocation points and $\boldsymbol{u}^{0}$ at the interior collocation points 
\item Solve $\boldsymbol{r}^0=\boldsymbol{J}[\boldsymbol{\alpha}^0\ \boldsymbol{\beta}^0]^T$ for
  $\boldsymbol{\alpha}^0$ and $\boldsymbol{\beta}^0$
\item evaluate the initial acceleration, $\boldsymbol{a}^0$, using eq. (\ref{Acceleration}) 
\item begin time stepping 
\begin{enumerate}
\item evaluate $\boldsymbol{u}^{t}=\boldsymbol{u}^{t-1} + 
                                  \boldsymbol{v}^{t-1}\Delta t + 
                                  \boldsymbol{a}^{t-1}\Delta t^2$
\item Form $\boldsymbol{r}^t$
\item Solve $\boldsymbol{r}^t=\boldsymbol{J}[\boldsymbol{\alpha}^t\ \boldsymbol{\beta}^t]^T$ for
  $\boldsymbol{\alpha}^t$ and $\boldsymbol{\beta}^t$
\item evaluate $\boldsymbol{a}^t$ from eq. (\ref{Acceleration})
\item evaluate $\boldsymbol{v}^t = \boldsymbol{v}^{t-1} +
  1/2(\boldsymbol{a}^{t-1} + \boldsymbol{a}^t)\Delta t$
\end{enumerate} 
\end{enumerate} 

There are, however, a few additional details that need to be
addressed.  First, we must deal with the material properties, $\mu$,
$\lambda$, and $\rho$.  We use the material properties from the
Preliminary Reference Earth Model (PREM) \citep{D1981}.  PREM was
inferred using observed P and S wave travel times in addition to other
geophysical data.  We then have a means of validating our numerical
solution because the travel times predicted by our numerical
simulation should be consistent with observable travel times.  The
material properties for PREM are given at discrete depths, and so we
perform an interpolation using multiquadratic RBFs to obtain PREM as a
continuous and analytically differentiable function.  Our interpolants
for PREM are shown in figure 2, 3, and 4.

\begin{figure}
\includegraphics[scale=0.45]{figures/Pvel}
\centering
\caption{P wave velocities ($\sqrt{(\lambda + 2\mu)/\rho}$) from PREM.
  Gray dots indicate collocation points used in solving
  eq. (\ref{DiffEq})}

\end{figure}
\begin{figure}
\includegraphics[scale=0.45]{figures/Svel}
\centering
\caption{S wave velocities ($\sqrt{\mu/\rho}$) from PREM}

\end{figure}
\begin{figure}
\includegraphics[scale=0.45]{figures/density}
\centering
\caption{Densities from PREM}

\end{figure}

We also must address how the collocation points are chosen.  PREM has
a major material discontinuity at the core-mantle boundary and we would
ideally prefer to have a higher density of nodes at that interface.
Additionally, the lower mantle has higher P and S wave velocities
relative to the upper mantle, crust, and core.  We would then prefer a
lower density of collocation points in the lower mantle in order to
keep our solution stable during time stepping.  Although we seek
regional scale variations in node density, we still want our
collocation points to have low discrepancy in any given region of the
earth.  We have developed a simple method to produce a distribution of
collocation points that satisfy our specifications.  Let
$\psi:\boldsymbol{x}\to [0,1]$ be a normalized function describing the
desired density of collocation points at position $\boldsymbol{x}$.
Let $H^1_k$ be the $k^{th}$ element in a one-dimensional Halton
sequence.  Let $H^2_k$ be the $k^{th}$ element of a two-dimensional
Halton sequence which has been scaled so that $H^2_k$ can lie anywhere
in the problem domain, $D$. The set of $N$ collocation points,
$\boldsymbol{C}$, with the desired density can then be found with algorithm 1.

In our experience, we found that it is necessary to have a small
buffer between the boundary collocation points and interior
collocation points.  This is essential for a stable solution.  We take
the width of the buffer to be the smallest distance between boundary
nodes.  The reader may be able to notice this slight buffer in either
Figures 2, 3 or 4.
\begin{algorithm}
\caption{Collocation points}
\begin{algorithmic}
\STATE $\boldsymbol{C} = \emptyset$
\STATE $k = 0$
\WHILE{(length of $\boldsymbol{C}) < N$}
  \IF{($\phi(H^2_k) > H^1_k) \& (H^2_k \in D)$}
    \STATE append $H^2_k$ to $\boldsymbol{C}$
  \ENDIF
  \STATE $k = k + 1$  
\ENDWHILE
\end{algorithmic}
\end{algorithm}
\section*{Results}
Our numerical solution to eq. (\ref{DiffEq}) is shown in figures 5-14.
The most prominent feature in our solution are the Raleigh waves which
travel along the surface and have decaying amplitude with depth.  One
can identify the P waves as wave fronts which have displacements
parallel to the propagation direction.  These are the faint blue, fast
traveling waves.  The S waves can be identified as having displacements
perpendicular to the propagation direction.  
%We can also be assured that the
%material properties are properly implemented because S waves, which
%have displacements parallel to the wave front, do not enter the core
%because it has a shear wave velocity of zero.

It is worth noting that the solution has symmetry about the vertical
axis.  This symmetry is no surprise and we have attempted to solve
eq. (\ref{DiffEq}) on only half the domain.  However, we have found
that solutions using RBFs are incredibly sensitive to corners in the
domain and our solution quickly went unstable after a few time steps.
Although we are being inefficient by solving for displacements on a
full disk, we are now able to check that our solution is accurate
based on the symmetry of our results.  With this means of validation,
it is apparent that our solution can not possibly be accurate beyond
about 800 seconds when substantial asymmetric structures appear in the
solution. This is unfortunate because seismic waves can be observed
for thousands of seconds after an earthquake.  We note that the
solution presented here required about $8 GB$ of memory and took 30
hours to run on two cores of a desktop computer (using a parallelized
matrix solver).  A higher density of collocation points is clearly
necessary to accurately model seismic wave propagation but we do not
have the computational resources for such a calculation.  In fact,
existing spectral element software for solving seismic wave
propagation on a similar two-dimensional domain is intended for use on
state of the art super computers \citep[e.g.][]{N2007}, so we are not
too discouraged by the poor accuracy we obtained using a standard
desktop computer.

\section*{Conclusion}
After our best effort, we were unable to satisfactorily model
seismic wave propagation using radial basis functions.  This is not
due to any fault of the RBF method, but rather due to our
computational limitations.  This is not to say that RBF methods have
no faults.  In our experience the RBF method seems quite precarious.
A poorly chosen shape parameter or a single collocation point that is
too close to the boundary would cause the entire solution to go
unstable.  Additionally, a corner in the problem domain can lead to an
unstable solution, even though the RBF method is touted as being able
to handle arbitrary geometries.  Further exploration is clearly needed
to make our model for seismic wave propagation more robust.

\section*{Additional notes}
All the software we have written to produce the results shown in this
paper are included below.  The software (with eventual bug fixes) can
also be found at https://github.com/treverhines/SeisRBF.  The packages
used in our software can be found in the Anaconda Python distribution
from Continuum Analytics.

\begin{thebibliography}{}
  \bibitem[Aki \& Richards(2002)]{AR2002} Aki, K. \& Richards, P.,
    2002. Quantitative Seismology, second edition, \textit{University
      Science Books}


  \bibitem[Driscoll \& Fornberg(2002)]{D2002} Driscoll, T. \& Fornberg, B.,
    2002. Interpolation in the limit of increasingly fiat radial basis
    functions, \textit{Computers and Mathematics with Applications},
    43, 413-422.

  \bibitem[Dziewonski \& Anderson(1981)]{D1981} Dziewonski, A. M. \&
    Anderson, D.L., 1981. Preliminary reference Earth
    model. \textit{Phys Earth Planet Inter}, 25,
    297-356. doi:10.1016/0031-9201(81)90046-7.

  \bibitem[Fichtner(2011)]{F2011} Fichtner, A., 2011. Full Seismic
    Waveform Modelling and Inversion, \textit{Springer}
 
  \bibitem[Nissen-Meyer et al.(2007)]{N2007} Nissen-Meyer, T.,
    Fournier, A., Dahlen, F. A., 2007. A 2-D spectral-element method
    for computing spherical-earth seismograms--I. Moment-tensor
    source, Geophys. J. Int., Vol. 168, issue 3, pp. 1067-1093
\end{thebibliography}

\begin{figure}
\includegraphics[scale=0.4]{figures/100s}
\centering
\caption{Solution for displacements at 100 seconds after an earthquake. Color
  indicates the magnitude of displacement and vectors indicate
  direction}
\end{figure}

\begin{figure}
\includegraphics[scale=0.4]{figures/200s}
\centering
\caption{Solution for displacements at 200 seconds after an earthquake}
\end{figure}

\begin{figure}
\includegraphics[scale=0.4]{figures/300s}
\centering
\caption{Solution for displacements at 300 seconds after an earthquake}
\end{figure}

\begin{figure}
\includegraphics[scale=0.4]{figures/400s}
\centering
\caption{Solution for displacements at 400 seconds after an earthquake}
\end{figure}

\begin{figure}
\includegraphics[scale=0.4]{figures/500s}
\centering
\caption{Solution for displacements at 500 seconds after an earthquake}
\end{figure}

\begin{figure}
\includegraphics[scale=0.4]{figures/600s}
\centering
\caption{Solution for displacements at 600 seconds after an earthquake}
\end{figure}

\begin{figure}
\includegraphics[scale=0.4]{figures/700s}
\centering
\caption{Solution for displacements at 700 seconds after an earthquake}
\end{figure}

\begin{figure}
\includegraphics[scale=0.4]{figures/800s}
\centering
\caption{Solution for displacements at 800 seconds after an earthquake}
\end{figure}

\begin{figure}
\includegraphics[scale=0.4]{figures/900s}
\centering
\caption{Solution for displacements at 900 seconds after an earthquake}
\end{figure}

\begin{figure}
\includegraphics[scale=0.4]{figures/1000s}
\centering
\caption{Solution for displacements at 1000 seconds after an earthquake}
\end{figure}

\break
\pythonscript{scripts/SeisRBF}{SeisRBF.py} 
\pythonscript{scripts/radial}{radial.py} 
\pythonscript{scripts/halton}{halton.py} 
\pythonscript{scripts/usrfcn}{usrfcn.py} 

\end{document}


